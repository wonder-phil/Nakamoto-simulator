\documentclass[12pt]{article}
\textwidth 15.4 cm
\textheight 21.5 cm
\topmargin 0.5cm
\evensidemargin 2 cm
\oddsidemargin 3 mm

\newcommand{\qed}{\rule[-0.2ex]{0.3em}{1.4ex}}
\newcommand{\out}[1]{\mbox{\bf {\sf Out}}({#1})}
\newenvironment{Proof}%
{\par\vspace{0.5ex}\noindent{\bf Proof:}\hspace{0.5em}}%
{\nopagebreak
\strut\nopagebreak
\hspace{\fill}\qed\par\medskip\noindent}
\newenvironment{proof}% lowercase version...
{\par\vspace{0.5ex}\noindent{\bf Proof:}\hspace{0.5em}}%
{\nopagebreak
\strut\nopagebreak
\hspace{\fill}\qed\par\medskip\noindent}
\newenvironment{proofattempt}%
{\par\vspace{0.5ex}\noindent{\bf Proof Attempt:}\hspace{0.5em}}%
{\nopagebreak
\strut\nopagebreak
\hspace{\fill}\qed\par\medskip\noindent}
\newtheorem{theorem}{Theorem}
\newtheorem{prop}{Proposition}
\newtheorem{property}{Property}
\newtheorem{lemma}{Lemma}
\newtheorem{fact}{Fact}
\newtheorem{definition}{Definition}
\newtheorem{corollary}{Corollary}
\newtheorem{remark}{Remark}
\newtheorem{observation}{Observation}
\newtheorem{conjecture}{Conjecture}

\newcommand{\support}{\mbox{\sf supp}}
\newcommand{\OMS}{\mbox{$\bf \Omega_{S}$}}


\usepackage{amsmath}
\usepackage{amsfonts}
\usepackage{color}
\usepackage{circuitikz}
\usepackage{listings}
%\usepackage{tikz}

\usepackage{listings}
\usepackage{color}
\usepackage{textcomp}

\lstdefinelanguage{JavaScript}{
  morekeywords=[1]{break, continue, delete, else, for, function, if, in,
    new, return, this, typeof, var, void, while, with},
  % Literals, primitive types, and reference types.
  morekeywords=[2]{false, null, true, boolean, number, undefined,
    Array, Boolean, Date, Math, Number, String, Object,new},
  % Built-ins.
  morekeywords=[3]{eval, parseInt, parseFloat, escape, unescape,let,var,const,await,async,try,catch,import,from,export},
  morekeywords=[4]{className,h1,h2,div,header,Card.Text,Card.Body},
  sensitive,
  morecomment=[s]{/*}{*/},
  morecomment=[l]//,
  morecomment=[s]{/**}{*/}, % JavaDoc style comments
  morestring=[b]',
  morestring=[b]"
}[keywords, comments, strings]

% Requires package: color.
\definecolor{mediumgray}{rgb}{0.3, 0.4, 0.4}
\definecolor{mediumblue}{rgb}{0.0, 0.0, 0.8}
\definecolor{forestgreen}{rgb}{0.13, 0.55, 0.13}
\definecolor{darkviolet}{rgb}{0.58, 0.0, 0.83}
\definecolor{royalblue}{rgb}{0.25, 0.41, 0.88}
\definecolor{crimson}{rgb}{0.86, 0.8, 0.24}
\definecolor{amber}{rgb}{1.0, 0.75, 0.0}
\definecolor{cobalt}{rgb}{0.0, 0.28, 0.67}
\definecolor{darkcandyapplered}{rgb}{0.64, 0.0, 0.0}

\lstdefinestyle{JSES6Base}{
  backgroundcolor=\color{white},
  basicstyle=\ttfamily,
  alsoletter={.},
  breakatwhitespace=false,
  breaklines=false,
  captionpos=b,
  columns=fullflexible,
  commentstyle=\color{mediumgray}\upshape,
  emph={},
  emphstyle=\color{crimson},
  extendedchars=true,  % requires inputenc
  fontadjust=true,
  frame=single,
  identifierstyle=\color{black},
  keepspaces=true,
  keywordstyle=\color{mediumblue},
  keywordstyle={[2]\color{darkviolet}},
  keywordstyle={[3]\color{royalblue}},
  keywordstyle={[4]\color{darkcandyapplered}},
  numbers=left,
  numbersep=5pt,
  numberstyle=\tiny\color{black},
  rulecolor=\color{black},
  showlines=true,
  showspaces=false,
  showstringspaces=false,
  showtabs=false,
  stringstyle=\color{forestgreen},
  tabsize=2,
  title=\lstname,
  upquote=true  % requires textcomp
}

\lstdefinestyle{JavaScript}{
  language=JavaScript,
  style=JSES6Base
}
\lstdefinestyle{ES6}{
  language=ES6,
  style=JSES6Base
}

\lstset{
	language=JavaScript
}


\lstdefinelanguage{bash}{
  keywords={cd, npm, npx,sudo, apt-get, ifconfig, mosquitto_pub, mosquitto_sub, ping},
  basicstyle=\small,
  alsoletter=?!-,
  alsodigit=\$\%&*+./:<=>@^_~,
  sensitive=false,
  morecomment=[l]{\#},
  morecomment=[s]{/*}{*/},
  morestring=[b]",
  basicstyle=\small\ttfamily,
  keywordstyle=\bf\ttfamily\color[rgb]{0,.3,.7},                                                                                                             
  commentstyle=\color[rgb]{0.133,0.545,0.133},
  stringstyle={\color[rgb]{0.75,0.49,0.07}},
  upquote=true,
  breaklines=true,
  breakatwhitespace=true,
  literate=*{`}{{`}}{1},
  keywords=[2]{myRed},
  keywordstyle=[2]\color{red}
}


\usepackage{textcomp}
\usepackage{booktabs}


\DeclareMathOperator*{\argmax}{arg\,max}
\DeclareMathOperator*{\argmin}{arg\,min}
\DeclareMathOperator*{\cov}{\sf cov}
 
\newcommand{\IR}{{\rm\hbox{I\kern-.15em R}}}
\newcommand{\reals}{{\rm\hbox{I\kern-.15em R}}}
\newcommand{\IN}{{\rm\hbox{I\kern-.15em N}}}
\newcommand{\IZ}{{\sf\hbox{Z\kern-.40em Z}}}
\newcommand{\id}{\mbox{\bf \em id}}
\newcommand{\lo}{\mbox{\bf \em loc}}
\newcommand{\E}{\mbox{\sf {I\kern-.15em E}}}
\newcommand{\F}{\mbox{\bf F}}
\newcommand{\Var}{\mbox{\bf Var}}
\newcommand{\Prob}{\mbox{\sf \hbox{I\kern-.15em P}}}
\newcommand{\g}{\mbox{\bf \em g}}
\newcommand{\SeqRank}{\mbox{\bf \em SequentialRank}}

\newcommand{\vat}{\mbox{\scshape \tiny VAT}}
\newcommand{\emm}{\mbox{\scshape \tiny EM}}
\newcommand{\ram}{\mbox{\scshape \tiny RAM}}
\newcommand{\mNC}{\ensuremath{\mbox{m}{\cal NC}^{1}}}

\newcommand{\ta}{\mbox{\color{blue} T}}
\newcommand{\h}{\mbox{\color{red} H}}

%
%
%
%\lstset{language=Scheme} 
%
%
%


\begin{document}
\title{A Dynamic {\bf Card} in React Bootstrap  - Part 2}
\author{
Phillip G. Bradford\thanks{phillip.bradford@uconn.edu, phillip.g.bradford@gmail.com,
{\sc University of Connecticut, Department of Computing, Stamford, CT USA}}
}

\date{\small\today}

\maketitle

%
%
%
\begin{abstract}
The Nakamoto-simulator dynamic card component uses a number of 
interesting features from React-bootstrap.
This dynamic card uses the {\em useState} function from {\em React hooks}.
\end{abstract}

%
%
%
%

%
%
\section{A dynamic card design}
\label{burn block dynamic card}


Listing~\ref{DynamicCardCollapsible} shows a basic React hook \lstinline|useState|.
The hook \lstinline|useState| remembers state through the life of the function \lstinline|DynamicCardCollapsible|.

%
%
\vspace{0.2in}
%
%

\begin{lstlisting}[label=DynamicCardCollapsible,style=JSES6Base, caption={Dynamic card}]
  export default function DynamicCardCollapsible ({ props }) {

  const [open, setOpen] = useState(false);
  const myList =  Object.entries(props);
  const firstThree = myList.slice(0, 3);
  const theRest = myList.slice(3);

  return (

    <div className="card custom-card border-primary mb-3">
    <Card style={{ maxWidth: "700px", margin: "auto" }} 
		  className="container ms-0" >
      <Card.Body>
        <Card.Title><h3>{props['card_title']}</h3>
		</Card.Title>
        <ul className="list-group list-group-flus no-bullets">
          <li className="list-group-item text-primary">
            <Button
              onClick={() => setOpen(!open)}
              aria-controls="collapse-text"
              aria-expanded={open}
              className="mb-2"
              >
              <h6>{open ? "Hash: " + props['hash'] :  
			              "Hash tail: " + props['hash'].slice(-20) }
			  </h6>
          </Button>
		  
		   <...>
		  );
};
\end{lstlisting}


%
%
%
%
\begin {thebibliography}{99}
%



\bibitem{Blake} Ryan J. Blake:
	      {\em Mastering Bootstrap 5},
	     Beginners Handbook,
       2024.

       
\bibitem{cards} React Bootstrap:
        {\em React Bootstrap Cards},\\
        {\em https://react-bootstrap.netlify.app/docs/components/cards/ },
       Accessed: 2025-03-15.

\end {thebibliography}
\end{document}

%
