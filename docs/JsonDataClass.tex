\documentclass[12pt]{article}
\textwidth 15.4 cm
\textheight 21.5 cm
\topmargin 0.5cm
\evensidemargin 2 cm
\oddsidemargin 3 mm

\newcommand{\qed}{\rule[-0.2ex]{0.3em}{1.4ex}}
\newcommand{\out}[1]{\mbox{\bf {\sf Out}}({#1})}
\newenvironment{Proof}%
{\par\vspace{0.5ex}\noindent{\bf Proof:}\hspace{0.5em}}%
{\nopagebreak
\strut\nopagebreak
\hspace{\fill}\qed\par\medskip\noindent}
\newenvironment{proof}% lowercase version...
{\par\vspace{0.5ex}\noindent{\bf Proof:}\hspace{0.5em}}%
{\nopagebreak
\strut\nopagebreak
\hspace{\fill}\qed\par\medskip\noindent}
\newenvironment{proofattempt}%
{\par\vspace{0.5ex}\noindent{\bf Proof Attempt:}\hspace{0.5em}}%
{\nopagebreak
\strut\nopagebreak
\hspace{\fill}\qed\par\medskip\noindent}
\newtheorem{theorem}{Theorem}
\newtheorem{prop}{Proposition}
\newtheorem{property}{Property}
\newtheorem{lemma}{Lemma}
\newtheorem{fact}{Fact}
\newtheorem{definition}{Definition}
\newtheorem{corollary}{Corollary}
\newtheorem{remark}{Remark}
\newtheorem{observation}{Observation}
\newtheorem{conjecture}{Conjecture}

\newcommand{\support}{\mbox{\sf supp}}
\newcommand{\OMS}{\mbox{$\bf \Omega_{S}$}}


\usepackage{amsmath}
\usepackage{amsfonts}
\usepackage{color}
\usepackage{circuitikz}
\usepackage{listings}
%\usepackage{tikz}

\usepackage{listings}
\usepackage{color}
\usepackage{textcomp}

\lstdefinelanguage{JavaScript}{
  morekeywords=[1]{break, continue, delete, else, for, function, if, in,
    new, return, this, typeof, var, void, while, with},
  % Literals, primitive types, and reference types.
  morekeywords=[2]{false, null, true, boolean, number, undefined,
    Array, Boolean, Date, Math, Number, String, Object,new},
  % Built-ins.
  morekeywords=[3]{eval, parseInt, parseFloat, escape, unescape,let,var,const,await,async,try,catch,import,from,export},
  morekeywords=[4]{className,h1,h2,div,header,Card.Text,Card.Body},
  sensitive,
  morecomment=[s]{/*}{*/},
  morecomment=[l]//,
  morecomment=[s]{/**}{*/}, % JavaDoc style comments
  morestring=[b]',
  morestring=[b]"
}[keywords, comments, strings]

% Requires package: color.
\definecolor{mediumgray}{rgb}{0.3, 0.4, 0.4}
\definecolor{mediumblue}{rgb}{0.0, 0.0, 0.8}
\definecolor{forestgreen}{rgb}{0.13, 0.55, 0.13}
\definecolor{darkviolet}{rgb}{0.58, 0.0, 0.83}
\definecolor{royalblue}{rgb}{0.25, 0.41, 0.88}
\definecolor{crimson}{rgb}{0.86, 0.8, 0.24}
\definecolor{amber}{rgb}{1.0, 0.75, 0.0}
\definecolor{cobalt}{rgb}{0.0, 0.28, 0.67}
\definecolor{darkcandyapplered}{rgb}{0.64, 0.0, 0.0}

\lstdefinestyle{JSES6Base}{
  backgroundcolor=\color{white},
  basicstyle=\ttfamily,
  alsoletter={.},
  breakatwhitespace=false,
  breaklines=false,
  captionpos=b,
  columns=fullflexible,
  commentstyle=\color{mediumgray}\upshape,
  emph={},
  emphstyle=\color{crimson},
  extendedchars=true,  % requires inputenc
  fontadjust=true,
  frame=single,
  identifierstyle=\color{black},
  keepspaces=true,
  keywordstyle=\color{mediumblue},
  keywordstyle={[2]\color{darkviolet}},
  keywordstyle={[3]\color{royalblue}},
  keywordstyle={[4]\color{darkcandyapplered}},
  numbers=left,
  numbersep=5pt,
  numberstyle=\tiny\color{black},
  rulecolor=\color{black},
  showlines=true,
  showspaces=false,
  showstringspaces=false,
  showtabs=false,
  stringstyle=\color{forestgreen},
  tabsize=2,
  title=\lstname,
  upquote=true  % requires textcomp
}

\lstdefinestyle{JavaScript}{
  language=JavaScript,
  style=JSES6Base
}
\lstdefinestyle{ES6}{
  language=ES6,
  style=JSES6Base
}

\lstset{
	language=JavaScript
}


\usepackage{textcomp}
\usepackage{booktabs}


\DeclareMathOperator*{\argmax}{arg\,max}
\DeclareMathOperator*{\argmin}{arg\,min}
\DeclareMathOperator*{\cov}{\sf cov}
 
\newcommand{\IR}{{\rm\hbox{I\kern-.15em R}}}
\newcommand{\reals}{{\rm\hbox{I\kern-.15em R}}}
\newcommand{\IN}{{\rm\hbox{I\kern-.15em N}}}
\newcommand{\IZ}{{\sf\hbox{Z\kern-.40em Z}}}
\newcommand{\id}{\mbox{\bf \em id}}
\newcommand{\lo}{\mbox{\bf \em loc}}
\newcommand{\E}{\mbox{\sf {I\kern-.15em E}}}
\newcommand{\F}{\mbox{\bf F}}
\newcommand{\Var}{\mbox{\bf Var}}
\newcommand{\Prob}{\mbox{\sf \hbox{I\kern-.15em P}}}
\newcommand{\g}{\mbox{\bf \em g}}
\newcommand{\SeqRank}{\mbox{\bf \em SequentialRank}}

\newcommand{\vat}{\mbox{\scshape \tiny VAT}}
\newcommand{\emm}{\mbox{\scshape \tiny EM}}
\newcommand{\ram}{\mbox{\scshape \tiny RAM}}
\newcommand{\mNC}{\ensuremath{\mbox{m}{\cal NC}^{1}}}

\newcommand{\ta}{\mbox{\color{blue} T}}
\newcommand{\h}{\mbox{\color{red} H}}

%
%
%
%\lstset{language=Scheme} 
%
%
%


\begin{document}
\title{{\bf JSON} data class for REST calls}
\author{
Phillip G. Bradford\thanks{phillip.bradford@uconn.edu, phillip.g.bradford@gmail.com,
{\sc University of Connecticut, Department of Computing, Stamford, CT USA}}
}

\date{\small\today}

\maketitle

%
%
%
\begin{abstract}
These notes discuss a flexible Javascript JSON data class: \lstinline|JsonContainer|.
\end{abstract}

%
%
%
%
\section{JsonContainer class}
\label{JsonContainer class}

This class is designed to be inherited from.
Such classes are common and similar or almost identical classes can be found all over.

%
%
\vspace{0.2in}
%
%

\begin{lstlisting}[label=JsonContainer,style=JSES6Base, caption={JsonContainer class}]
class JsonContainer {
    constructor(data = {}) {
        this.setData(data);
    }

    setData(data) {
        if (typeof data !== "object" || data === null) {
            throw new Error("Data must be a valid JSON object");
        }
        this._data = { ...data }; // Clone 
    }

    getData() {
        return { ...this._data };
    }

    getField(key, defaultValue = null) {
        return this._data.hasOwnProperty(key) 
		          ? this._data[key] : defaultValue;
    }

    setField(key, value) {
        this._data[key] = value;
    }

    hasField(key) {
        return this._data.hasOwnProperty(key);
    }

    toJson() {
        return JSON.stringify(this._data, null, 2);
    }
}
\end{lstlisting}



%
%
\vspace{0.2in}
%
%

%
%
%
\begin{lstlisting}[label=JsonContainer-example,style=JSES6Base, caption={JsonContainer example}]
const apiResponse = {
    id: 123,
    name: "Sample Data",
    attributes: {
        color: "red",
        size: "large"
    }
};

// Initialize container
const container = new JsonContainer(apiResponse);

console.log(container.getField("name"));
console.log(container.getField("unknown", "Default Value"));

container.setField("status", "active");
console.log(container.getData());

// Merge additional data
container.mergeData({ price: 99.99, available: true });

console.log(container.toJson());
\end{lstlisting}


%
%
\vspace{0.2in}
%
%

Listing~\ref{JsonContainer-example} shows a basic implementation of this the class JsonContainer.

%
%
\section{Notes}
%
%

ES6 (ECMAScript 6) is ECMAScript 2015.

React is based on ES6.


%
%
%
%
\begin {thebibliography}{99}
%


\bibitem{All-ECMA-JS}
https://tc39.es/ecma262/



\end {thebibliography}
\end{document}

%
